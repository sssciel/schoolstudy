\documentclass{article}

\usepackage[colorlinks=true, allcolors=blue]{hyperref}
\usepackage{amsmath}
\usepackage{amssymb}
\usepackage[english, russian]{babel}
\usepackage[letterpaper,top=2cm,bottom=2cm,left=3cm,right=3cm,marginparwidth=1.75cm]{geometry}
\usepackage{graphicx}
\usepackage[colorlinks=true, allcolors=blue]{hyperref}
\setlength{\parindent}{0pt}
\setlength{\parskip}{1em}  

\newcounter{problem}[section]
\renewcommand{\theproblem}{\arabic{problem}}
\newcommand{\problem}[1][]{%
    \stepcounter{problem}%
    \textbf{Задача~\theproblem#1.}
}

\title{8 класс. Математическая вертикаль}
\date{}

\begin{document}
\maketitle
\tableofcontents

\section{Сравнение чисел}

\problem{} Какое из двух чисел больше:
\[
\dfrac{41}{99} \quad \text{или} \quad  \dfrac{411}{991}?
\]

\problem{} Сравните дроби $\dfrac{222\,221}{222\,222}$, $\dfrac{333\,332}{333\,334}$ и $\dfrac{444\,442}{444\,445}$, расположите их в порядке возрастания.

\problem{} Расположите в порядке возрастания числа: $333^{3},\ 3^{333},\ 33^{33}$.

\problem{} Найдите наибольшее натуральное $n$, при котором $n^{200} < 4^{300}$.

\problem{} Какое из двух чисел больше:
\[
1000^{100} \quad \text{или} \quad  500^{50} \cdot 1500^{50}?
\]

\problem{} Какое из двух чисел больше:
\[
\sqrt[3]{\dfrac{2024}{2025}} \quad \text{или} \quad \sqrt[3]{\dfrac{2025}{2026}}?
\]

\problem{} Какое из двух чисел больше:
\[
\sqrt[3]{4} + \sqrt{2} \quad \text{или} \quad \sqrt[3]{3}?
\]

% v2
\problem{} Какое из двух чисел больше: 
\[
1 \quad \text{или} \quad  \dfrac{32}{97} + \dfrac{70}{211} + \dfrac{146}{439}
\]

\problem{} На каком из описанных ниже интервалов, разбивающих числовую ось, лежит число 0?
\[
x^3 < y^8 < y^3 < x^{12},
\]

\problem{} Какое из двух чисел больше: 
\[
2025^{2025} + 2023^{2023} \quad \text{или} \quad  2025^{2023} + 2023^{2025}?
\]

\problem[*] Какое из двух чисел больше:
\[
\dfrac{100}{101} \times \dfrac{102}{103} \times \ldots \times \dfrac{1022}{1023} \quad \text{или} \quad \dfrac{5}{16}.
\]

\problem[**] Какое из двух чисел больше:
\[
\sqrt{2016 + \sqrt{2015 + \sqrt{2016}}} \quad \text{или} \quad \sqrt{2015 + \sqrt{2016 + \sqrt{2015}}} \, ?
\]

\newpage
\section{Неравенства}
\problem{} 
Оцените площадь и периметр, которые может иметь прямоугольник, если одна его сторона может иметь длину от 20 до 30 см, а другая — от 50 до 60 см.

\problem{}
Пусть переменные $x$ и $y$ удовлетворяют неравенствам
\[
-0{,}9 < x < 2{,}5, \quad -3 < y < -2.
\]
При этом известно, что значение дроби
\[
\dfrac{1{,}1 + x}{y}
\]
является целым числом. Определите это целое число.

\end{document}
